\documentclass[a4paper,11pt]{article}

% Packages nécessaires
\usepackage[utf8]{inputenc}
\usepackage[T1]{fontenc}
\usepackage{geometry}
\geometry{margin=1in}
\usepackage{graphicx}
\usepackage{booktabs}
\usepackage{enumitem}
\usepackage{xcolor}
\usepackage{hyperref}
\usepackage{longtable}
\usepackage{titlesec}
\usepackage{fancyhdr}
\usepackage{ifthen} % Nécessaire pour \ifthenelse

% Couleurs pour les niveaux de danger
\definecolor{normalgreen}{RGB}{40,167,69}
\definecolor{dangerousyellow}{RGB}{255,193,7}
\definecolor{hyperdangerousred}{RGB}{220,53,69}

% Configuration des en-têtes et pieds de page
\pagestyle{fancy}
\fancyhf{}
\fancyhead[L]{Rapport Sécurité Urbaine}
\fancyhead[R]{ID: {{ report.id }}}
\fancyfoot[C]{\thepage}

% Format des sections
\titleformat{\section}{\Large\bfseries}{\thesection}{1em}{}
\titleformat{\subsection}{\large\bfseries}{\thesubsection}{1em}{}

% Police par défaut
\usepackage{times}

\begin{document}

% Page de titre
\begin{titlepage}
    \centering
    \vspace*{2cm}
    {\Huge\bfseries Rapport d'Analyse de Sécurité Urbaine \par}
    \vspace{1cm}
    {\Large Rapport ID: {{ report.id }} \par}
    \vspace{0.5cm}
    {\large Nom du Rapport: {{ report.name }} \par}
    \vspace{0.5cm}
    {\large Date de Génération: {{ current_date }} \par}
    \vspace{2cm}
    {\large Préparé par: Système de Sécurité Urbaine \par}
    \vfill
\end{titlepage}

% Résumé des statistiques
\section{Résumé des Statistiques}
Les statistiques globales du rapport sont les suivantes :

\begin{itemize}[leftmargin=*]
    \item \textcolor{normalgreen}{\textbf{Normale}}: {{ stats.normal }} détections
    \item \textcolor{dangerousyellow}{\textbf{Dangereuse}}: {{ stats.dangerous }} détections
    \item \textcolor{hyperdangerousred}{\textbf{Hyperdangereuse}}: {{ stats.hyperdangerous }} détections
\end{itemize}

% Catégories détectées
\section{Catégories Détectées}
Les catégories d'objets détectés dans ce rapport, avec leur nombre d'instances, sont listées ci-dessous :

\begin{table}[h]
    \centering
    \begin{tabular}{lc}
        \toprule
        \textbf{Catégorie} & \textbf{Nombre d'Instances} \\
        \midrule
        
        {{ category }} & {{ count }} \\
        
        \bottomrule
    \end{tabular}
    \caption{Répartition des catégories détectées}
\end{table}

% Détails des détections
\section{Détails des Détections}
Les détections individuelles sont détaillées ci-dessous :

\begin{longtable}{p{0.2\textwidth}p{0.3\textwidth}p{0.2\textwidth}p{0.25\textwidth}}
    \toprule
    \textbf{Niveau de Danger} & \textbf{Date/Heure} & \textbf{Objets Détectés} & \textbf{Validation} \\
    \midrule
    \endhead
    
    \textbf{
        \ifthenelse{\equal{ {{ detection.danger_level|default:"NORMAL" }} }{HYPERDANGEROUS}}%
        {\textcolor{hyperdangerousred}{Hyperdangereuse}}%
        {\ifthenelse{\equal{ {{ detection.danger_level|default:"NORMAL" }} }{DANGEROUS}}%
            {\textcolor{dangerousyellow}{Dangereuse}}%
            {\textcolor{normalgreen}{Normale}}}%
    } \newline
    (Simulé)
    &
    {{ detection.detection_timestamp|date:"d/m/Y H:i" }} &
    
        
            {{ obj.category }} ({{ obj.confidence|floatformat:2 }})
            , 
        
            Inconnu
        
    
        Aucun
    
    &
    
        \ifthenelse{\equal{ {{ detection.validation.is_correct|stringformat:"s" }} }{True}}%
        {Correcte}%
        {Incorrecte ({{ detection.validation.corrected_category|default:"Non spécifié" }})}
    
        Non validé
    
    \\
    \midrule
    
\end{longtable}

\end{document}